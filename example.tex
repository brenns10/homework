\documentclass{homework}
\student{Eddie Maldonado}
\course{Math XYZ}
\assignment{Problem Set 5}
\duedate{October 9, 2012}
\usepackage{amsmath}
\begin{document}
\maketitle


\problemsection{Page XYZ}

\begin{problem}{3}
  This is an example problem. It is here to serve as a placeholder in
  order to demonstrate the capabilities of this \LaTeX{} document
  class.

  Note that you can still do paragraph breaks within problems.
\end{problem}

\begin{problem}{4}
  This would presumably be the next problem.
\end{problem}

\problemsection{Page ABC}

\begin{problem}{10}
  This is a problem form a different section or source.
\end{problem}

\begin{problem}{11}
  \begin{question}
    What is $2 + 2$?
  \end{question}
  The answer is $4$. Note that this problem has its statement
  included. Note that, if you begin a new paragraph after issuing the
  \verb+\statement+ command, it will be indented. If you do not want
  this effect, keep the beginning of the problem as part of the same
  paragraph.
\end{problem}

\newpage

\begin{problem}{17}
  Here's yet another problem. I've manually inserted a page break in
  order to demonstrate the appearance of the header on pages after the
  first.
\end{problem}

\begin{problem}{25}
  \begin{question}
    This is a demonstration of the question environment.  It is basically the
    same as the statement command, but it is an environment.  The advantage to
    an environment is that many LaTeX tools cannot cope with math within the
    statement command argument, but they can cope with it within an environment.

    $\int_0^1 x^2 dx$

  \end{question}

  So, you have the option to use the question environment for long question
  statements that have a lot of math.
\end{problem}
\end{document}